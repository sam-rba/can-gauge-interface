\documentclass{article}
\usepackage{graphicx}

\title{
Analog Gauge Driver with CAN Interface \\
\large COMP490 Project Proposal
}
\author{
Sam Anthony 40271987 \\
sam@samanthony.xyz \\ s\_a365@concordia.ca
\and
Hovhannes Harutyunyan, PhD \\
Department of Computer Science and Software Engineering \\
haruty@encs.concordia.ca
\and
Concordia University \\
}

\begin{document}

\maketitle

Installing aftermarket gauges in a car typically requires installing sensors as well.
However, such sensors are often already present, and are used by the ECU (engine control unit).
Thus, the installation of aftermarket gauges can result in duplicate sensors which add complexity without augmenting the functionality or reliability of the vehicle.
Sensors installed like this are not redundant.
In fact, they reduce reliability, because each is a single point of failure.

The proposed device allows gauges to use the sensors already on the car.
It retrieves sensor data from the ECU via the CAN bus (controller area network bus) and transforms the data into a format that the gauges can understand: a 0--5V analog signal in the case of a temperature or pressure gauge, or a square wave in the case of a tachometer or speedometer.

The device is an embedded system comprising a microcontroller, a CAN controller and transceiver, several DACs (digital-to-analog converters), and PROM (programmable read-only memory).
The CAN interface is used for retrieving data from the ECU via the bus.
The DACs drive analog signals to the temperature and/or pressure gauges.
The microcontroller has an integrated PWM peripheral for driving a square wave to the tachometer and/or speedometer.
The PROM stores the calibration: a table that maps CAN data values to voltages or frequencies.
The microcontroller has a USB interface for programming the PROM from a computer.

\begin{figure}
	\includegraphics[width=\textwidth]{diagram.png}
	\caption{System diagram}
\end{figure}

The project has three parts: hardware design, software development, and testing.
The hardware and software development can be carried out concurrently.
Testing is the final step.

Hardware design involves selecting ICs (integrated circuits), creating a circuit schematic, and designing a PCB (printed circuit board).
Once the board design is finalized, it can be sent for manufacturing.

Two pieces of software must be written.
The first runs on the microcontroller.
Essentially, it must communicate with the various peripherals by transforming and transferring data between them.
It must fetch frames from the CAN controller and decipher them.
The CAN data are used to lookup the output value in the ROM.
Either the PWM peripheral or a DAC is used to send the appropriate signal to the gauge.
The microcontroller uses SPI (serial peripheral interface) to communicate with the peripherals.

The second piece of software runs on the user's computer.
It programs the PROM with calibration data.
It communicates with the microcontroller using a simple text-based protocol over USB.

\end{document}
