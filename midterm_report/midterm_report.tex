\documentclass{article}
\usepackage{graphicx}
\usepackage{hyperref}
\usepackage[backend=biber]{biblatex}
\usepackage{amsmath}

\title{\textsc{Comp} 490 Mid-Term Report}

\author{Sam Anthony 40271987 \\
sam@samanthony.xyz \\ s\_a365@concordia.ca
\and
Hovhannes Harutyunyan, PhD \\
Department of Computer Science and Software Engineering \\
haruty@encs.concordia.ca
\and
Concordia University \\
}

\addbibresource{references.bib}

\begin{document}

\maketitle
\tableofcontents
\pagebreak


\section{Project introduction}

The goal of the project is to build an electronic device for use in cars: it is an interface between the car's CAN bus (controller area network) \cite{can20b}, and some analog gauges installed in the cockpit.
An overview of the system is shown in figure \ref{fig:system}.

\begin{figure}
	\centering
	\includegraphics[width=\textwidth]{"../proposal/diagram.png"}
	\caption{system diagram}
	\label{fig:system}
\end{figure}


\section{Desiderata}

The device must be able to perform certain functions.
As well, there are some desirable properties that it should fulfil.

These function and desirable properties are as follows:

\begin{enumerate}
	\item{Receive standard and extended CAN frames from the bus.}
	\item{Decode information in the frames.}
	\item{Generate four analog 0--5V signals suitable for driving temperature or pressure gauges.}
	\item{Generate two variable-frequency square waves for a tachometer and a speedometer.}
	\item{Be user-programmable for any encoding scheme and gauge combination.}
	\item{Run on a 12V automotive electrical power supply.}
	\item{Operate reliably in an automotive environment: resist heat, vibration, and EMI (electromagnetic interference).}
\end{enumerate}


\section{Component selection}

A car is a harsh environment for an electronic device.
The device is subject to large variations in temperature, vibration, and EMI.
To increase reliability, AEC-certified parts were chosen wherever possible.


\subsection{Logic control}

The microcontroller is at the heart of the design.
A Microchip PIC16F1459 was chosen because of its simplicity, robustness, feature set, and low cost \cite{pic16f1459}.
It is an 8-bit microcontroller that features a USB peripheral, an SPI peripheral for communicating with the other ICs, and timers for waveform generation.
The PIC is a proven design that Microchip recommends for automotive applications.
It is available in a DIP package, making it convenient for prototyping on a breadboard (fig. \ref{fig:pic}).

\begin{figure}
	\centering
	\includegraphics[width=0.5\textwidth]{"pic16f1459.png"}
	\caption{Microchip PIC16F1459 8-bit microcontroller.}
	\label{fig:pic}
\end{figure}

A Microchip MCP2515 serves as the CAN controller \cite{mcp2515}.
It supports CAN 2.0B and it has an SPI interface for communicating with the PIC.
An MCP2561 transceiver goes along with it \cite{mcp2561}.
Like the PIC, both these chips are available in DIP packages for prototyping on the breadboard.


\subsection{Data storage}

The EEPROM is used to store the configuration.
This includes the encoding scheme that defines how parameters are encoded in CAN frames, as well as a table mapping parameter values to output signal values.

There are six such tables: one for each gauge.
Each table has 32 entries, and the mapping is from 16-bit word to 16-bit word.
Thus, the required size is $6 \times 32 \times 16 \times 2 = 6144$ bits, or 768 bytes.
Additionally, the encoding schemes will take a handfull of bytes per gauge.

a Microchip 25LC160C EEPROM was selected.
Its 16Kib (2KiB) of space is more than adequate to hold the configuration.


\subsection{Input/output}

The PIC has an integrated USB peripheral for communicating with a host computer.
The configuration is sent to the PIC via USB and stored on the EEPROM.

Four DACs (digital-to-analog converters) generate analog signals to drive the four pressure or temperature gauges.
Based on the characteristics of commonly-used pressure and temperature sensors \cite{bosch_pst}, it was determined that a resolution of 15mV/step was required.
Given the operating voltage of 5V, this meant that the DACs must have at least $5\text{V}/15\text{mV} \approx 333$ steps of resolution.
Thus, an 8-bit DAC with 256 steps would be insufficient, and so a 10-bit DAC was selected: namely a Microchip\footnote{It is purely a coincidence that all the ICs ended up being Microchip parts. I don't have any particular affinity to the company. It just so happens that they make all the right chips for this particular application.} MCP4912.

The MCP4912 incorporates two DACs in a single chip, so there are two chips per board.


\subsection{Power supply}

The ICs require a 5V supply.

A 12V automotive electrical system operates between 9--16V, with a nominal voltage of $\sim$13.7V.
The voltage ripple is often quite significant as well.
Thus, the power supply must be very robust to supply a stable voltage to the ICs.

The voltage drop $V_\text{Drop} = V_\text{In} - V_\text{Out}$ is $16\text{V} - 5\text{V} = 11\text{V}$ in the worst case.
This ruled out the use of a linear regulator, since it would dissipate too much power.
The power dissipation of a linear regulator is linear in $V_\text{Drop}$:

\begin{equation}
	P = (V_\text{In} - V_\text{Out}) \times I
\end{equation}

The load current is estimated to be $\le 250$mA \cite{power_budget}.
That means a linear regulator would dissipate up to $11\text{V} \times 0.250\text{A} = 2.75$W.
That amount of power from a tiny chip would be difficult to cool.
Thus, a switching regulator is the right choice for this design.

The downside of a switching regulator is that it produces a lot of noise in the PDN (power distribution network).
To isolate the other components from this noise, a two-stage PDN is used.
The first stage is the switching regulator itself, also known as a buck converter.
The buck drops the voltage from 12V down to 7V.

The second stage is composed of two linear regulators: one for the digital circuitry, and one for the analog circuitry.
Just like a buck converter, switched digital ICs introduce noise into the PDN.
Therefore, it is good practice to keep the digital and analog components separate.
The linear regulators drop the voltage from 7V down to the final 5V that the ICs require.
This second stage isolates the ICs from the noisy buck converter, and splitting the stage between two regulators keeps the digital and analog circuits isolated from one-another.

The buck converter is a Texas Instruments TPS5430 \cite{tps5430}.
It is surrounded by a couple of LC and RC networks to regulate the voltage and to dampen the output ripple.
See the datasheet and \cite{power_supply} for the process of selecting the accompanying passive components.
The linear regulators are ST L78M05ABs \cite{l78m}.

This design is certainly overkill for the application, but it is better than an under-developed PDN that could cause brown-outs during transient loads and variations in the supply voltage.


\section{PCB design and manufacture}

KiCad was used to design the schematic (fig. \ref{fig:schematic}) and the PCB (figs. \ref{fig:pcb_pours} \& \ref{fig:pcb_3d}).
JLCPCB was chosen to manufacture the printed circuit board.
At the time of writing, the board design has been finalized and submitted to JLC for manufacturing.
It should arrive any day now.

\begin{figure}
	\centering
	\includegraphics[width=\textwidth]{"schematic-v0.2.pdf"}
	\caption{schematic}
	\label{fig:schematic}
\end{figure}

\begin{figure}
	\centering
	\includegraphics[width=\textwidth]{"pcb_pours-v0.2.pdf"}
	\caption{PCB front and back copper pours, and drill holes}
	\label{fig:pcb_pours}
\end{figure}

\begin{figure}
	\centering
	\includegraphics[width=\textwidth]{"pcb_3d-v0.2.png"}
	\caption{PCB 3D render}
	\label{fig:pcb_3d}
\end{figure}

The board is a 4-layer design that uses a combination of surface-mount (SMD) and through-hole (THT) components.
The top and bottom layers are for signals, and the two middle layers are solid ground planes.
This ensures that all signal traces' fields are tightly coupled to ground directly above or below.
Power is routed on the bottom layer.

As mentioned above, most of the ICs are available in DIP (through-hole) packages, and I have been using them for prototyping on the breadboard.
Once the PCB arrives, I can transplant the chips into the board, along with the passive components.

The power supply parts, on the other hand, are SMD.
It is important to minimize loop distances in power supply circuits.
That is why power regulator chips are generally only available in smaller SMD packages.
The SMD components will be assembled by JLC, as I have neither the equipment nor the skill for SMD soldering.

The board was layed out with PCB design best-practices in mind.
Traces are widely-spaced to reduce coupling.
All signal vias are accompanied by a ground via to keep the E fields from spreading in the dielectric.
All traces are microstripped above a solid ground plane, again to keep the fields tight and to give the current a return path.
The noisy switching regulator is placed far away from the other components to reduce EMI---the sensitive analog signals and the DACs are far away, on the opposite side of the board.


\section{Firmware}

Firmware is the program that runs on the PIC microcontroller.
It is responsible for interacting with the peripherals and transforming data from the CAN bus into output signals for the gauges.

The firmware is written in C for Microchip's XC8 compiler.

Each peripheral---MCP2515, 25LC160C, and MCP4912---has a corresponding translation unit.
So far, I have written and tested the MCP4912 DAC unit.
The 25LC160C and MCP2515 units are mostly written but still require more testing.

The firmware must also communicate with a PC via USB.
The Microchip MLA USB library is employed for this purpose \cite{mla_usb}.
However, the USB code uses a lot of the PIC's flash memory.
Some system test builds already fail to link due to lack of space.
As a result, USB may have to be dropped from the project, and another method will have to be devised for programming the user configuration into the EEPROM.
This is a major disappointment, but I will have to work around it seeing as the hardware has already been finalized.


\section{Software}

Software is any program that runs on a host PC, as opposed to on the microcontroller.

I wrote a program called \texttt{usbcom} that communicates with the PIC via USB.
It connects its standard input to the OUT USB endpoint, and its standard output to the IN USB endpoint.
It is written in Go, and uses libusb.

\texttt{usbcom} and the PIC firmware communicate using a proprietary text-based protocol over the USB CDC (communications device class) interface.
The protocol is defined in loose Backus-Naur form \cite{protocol}.

I also wrote a Python script---\texttt{bittiming.py}---to calculate CAN bit timing parameters for the MCP2515.


\section{Next steps}

When the board arrives, I will do a visual inspection to ensure that nothing went wrong during manufacturing.
Then I will solder the THT components in place.
Once that is done, I will test the board for continuity and shorts.
With a fully assembled board, I can flash the firmware and migrate system testing from the breadboard to the printed prototype board.
I will finish developing the rest of the firmware, and run through the full suite of system tests that I will have at that point.

I bought some equipment for testing and debugging the system.
A USBtin USB/CAN interface will allow me test the reception of CAN frames \cite{usbtin}.
The EspoTek Labrador is a combined power supply, oscilloscope, and logic analyzer \cite{espotek_labrador}.
It will be used for powering the board and debugging the SPI lines.

If I have time, I would also like to do some experiments with the power supply to see how it handles changes in input voltage.


\printbibliography

\end{document}
