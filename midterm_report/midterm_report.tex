\documentclass{article}
\usepackage{graphicx}
\usepackage{hyperref}
\usepackage[backend=biber]{biblatex}

\title{\textsc{Comp} 490 Midterm Report}

\author{Sam Anthony 40271987 \\
sam@samanthony.xyz \\ s\_a365@concordia.ca
\and
Hovhannes Harutyunyan, PhD \\
Department of Computer Science and Software Engineering \\
haruty@encs.concordia.ca
\and
Concordia University \\
}

\addbibresource{references.bib}

\begin{document}

\maketitle


\section{Project introduction}

The goal of the project is to build an electronic device for use in cars: it is an interface between the car's CAN bus (controller area network) \cite{can20b}, and some analog gauges installed in the cockpit.
An overview of the system is shown in figure \ref{fig:system}.

\begin{figure}
	\centering
	\includegraphics[width=\textwidth]{"../proposal/diagram.png"}
	\caption{system diagram}
	\label{fig:system}
\end{figure}


\section{Desiderata}

The device must be able to perform certain functions.
As well, there are some desirable properties that it should fulfil.

These function and desirable properties are as follows:

\begin{enumerate}
	\item{Receive standard and extended CAN frames from the bus.}
	\item{Decode information in the frames.}
	\item{Generate four analog 0--5V signals suitable for driving temperature or pressure gauges.}
	\item{Generate two variable-frequency square waves for a tachometer and a speedometer.}
	\item{Be user-programmable for any encoding scheme and gauge combination.}
	\item{Run on a 12V automotive electrical power supply.}
	\item{Operate reliably in an automotive environment: resist heat, vibration, and EMI (electromagnetic interference).}
\end{enumerate}


\section{Component selection}

TODO


\section{PCB design and manufacture}

TODO


\section{Firmware}

TODO


\section{Software}

TODO


\section{Errata}

TODO


\section{Next steps}

TODO


\printbibliography

\end{document}
